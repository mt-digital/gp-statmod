\section{Method}\label{method}

A group polarization replication is proven undecidable, and therefore not
truly a replication, if we can find a model of \emph{mere agreement} that
appears to be polarized under that replication's experimental design. If we
cannot find such a model, we cannot determine a replication to be
undecidable. Mere agreement in this model is a dynamic where agreement
increases within the group, but extremism does not—in other words, A
\emph{null-polarization distributions} is a set of two Gaussian
distributions: one representing the pre-discussion opinions of the group, and
the other representing the post-discussion opinions. Our
\emph{null-polarization distributions} represent the process of the group
finding \emph{mere agreement}, which is when the average group opinion is
constant over time, but opinion variance decreases to represent agreement
(or, equivalently, \emph{consensus}~\cite{DeGroot1974}).

Null-polarization  We thus model opinions at each time, $t=\mathrm{pre}$ and
$t=\mathrm{post}$, as random variables drawn from a normal distribution with
mean $\mu_t$ and standard deviation $\sigma_t$.  we drop the time 

The null-polarization distributions are represented by 

in a simple, general way—i.e., a decrease in opinion variance but a static
opinion mean—we represent . In the general group polarization hypothesis, the
group is assumed to start out mostly agreeing in principle (i.e., mostly all
agree or disagree) but to different extremes. After discussion

Showing a replication is undecidable is sufficient to undermine it. However,
it does not tell us what is the probability that a particular experimental
design produces false positives, due to clipping or any other problem. 

first develop a mathematical model of estimate the best-case scenario false
positive rate for each experimental design,


\subsection{Mathematical Model of Null-Polarization Measurement Artifacts}

Latent opinions are modeled as normally-distributed random variables.  
This is a simple, sufficiently general way to represent group-level changes
in opinions as encoded in the general hypothesis of group polarization, that
likeminded groups tend to become more extreme on average as they discuss.
For time, $t$, then, we represent each participant $i$'s opinion as 
\begin{equation} 
  \omega_{i,t} \sim \mathcal{N}(\mu_t, \sigma_t).
  \label{eq:opinionDistribution} 
\end{equation} 
\noindent 

Group-level opinion change, then, can be represented as a change in opinion
mean, opinion variance, or both. Group polarization occurs 
when a group's average opinion increases over time. When likeminded people
interact, they tend to come to agree more over time—they \emph{find consensus}. 
By definition this means opinion variance decreases, 
whether the average changes or not.

We can define a null-polarization model, then, as one where the group's
average opinion stays constant, but the variance decreases: $\sigmapost
\leq \sigmapre$. Since the mean opinion is constant we just call it 
$\mu = \mupre = \mupost$. For conciseness, we define a null-polarization
model as a 3-tuple, $\nu = (\mu, \sigmapre, \sigmapost)$. 

\begin{table}[h]
  \caption{\textbf{Latent psychological variables and distributions.}}
  \begin{tabular}{cll} 
  \toprule
    \textbf{Variable} & \textbf{Description} & \textbf{Values} \\
  \midrule  
    $\omega_{i,t}$ & Latent opinion of individual $i$ at time $t$ & $\mathbb{R}$ \\
    $\mu_t$ & Mean latent opinion at time $t$ (constant in null-polarization) & $\mathbb{R}$ \\
    $\sigma_t$ & Variance of observed opnions at time $t$ & $\mathbb{R}$ \\
  \bottomrule
  \end{tabular} 
\end{table}

\begin{table}[h]
  \caption{\textbf{Experimental design and numerical model parameters.}}
  \begin{tabular}{cll} 
  \toprule
    \textbf{Variable} & \textbf{Description} & \textbf{Values} \\
  \midrule  
    $B$ & Number of bins & $5,6,7,\ldots$ \\
    $\theta_b$ & The $B+1$ threshold values that separate bins & 
      e.g., $-\infty, -2.5, -1.5,\ldots,2.5, \infty$ \\
    $b$ & Bin index—there are $B+1$ to mark $B$ bins & $0,1,2,\ldots,B$ \\
  \bottomrule
  \end{tabular} 
\end{table}

\begin{table}[h]
  \caption{\textbf{Simulated observed variables.}}
  \begin{tabular}{cll} 
  \toprule
    \textbf{Variable} & \textbf{Description} & \textbf{Values} \\
  \midrule  
    $\meanobst$ & Mean of all reported opinions (Eq. \ref{eq:meanobs}) & $\mathbb{R}$ \\
    $g$ & Observed group opinion polarization (Eq. \ref{eq:group-polarization})
        & $\mathbb{R}$ \\
    $d$ & Cohen's $d$ used as significance threshold 
        & $\mathbb{R}$ \\
    $\alpha_e,\alpha_s,\alpha_{\text{all}}$ & Family-wise/Type I error rate
    for an experiment, study, or across all experiments
             & $[0, 1]$ \\
    FDR & False discovery rate
        & $[0, 1]$ \\
  \bottomrule
  \end{tabular} 
\end{table}


\subsubsection{Simulated opinion measurements}

We simulate the collection of an \emph{ordinal distribution} of \emph{ordinal
observations}, as was performed in a prototypical group polarization experiment 
by integrating the latent opinion probability density function, $f(\omega)$,
over each bin's corresponding range of opinion values. For example, consider an
ordinal scale that runs from -3 (strongly disagree) to +3 (strongly agree)
(as used by \textcite{Moscovici1969}, for example).
the frequency with which the ``2'' bin is chosen is just the total
area underneath the probability density curve between 1.5 and 2.5.

Formally, the probability density function is
\begin{equation}
  f(\omega) = 
    \frac{1}{\sigma \sqrt{2 \pi}} e^{-\frac{(\omega - \mu)^2}{2 \sigma^2}}.
  \label{eq:normal_pdf}
\end{equation}
\noindent
Assume there are $B$ bins, meaning there are $B+1$ \emph{thresholds} that
define which latent opinions get mapped to which bin values. The bin value,
$b_i$, is the numerical value of the $i$th bin. 

\subsection{Root-finding Algorithm to Find Null-Polarization Models with Artifacts}


\subsection{Corpus Analysis to Determine Undecidable Replications}


\subsection{Evaluating Grouop Polarization Experimental Designs}


