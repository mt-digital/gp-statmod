\section{Method}\label{method}

A group polarization replication is proven undecidable, and therefore not
truly a replication, if we can find a model of \emph{mere agreement} that
appears to be polarized under that replication's experimental design. If we
cannot find such a model, we cannot determine a replication to be
undecidable. Mere agreement in this model is a dynamic where agreement
increases within the group, but extremism does not—in other words, A
\emph{null-polarization distributions} is a set of two Gaussian
distributions: one representing the pre-discussion opinions of the group, and
the other representing the post-discussion opinions. Our
\emph{null-polarization distributions} represent the process of the group
finding \emph{mere agreement}, which is when the average group opinion is
constant over time, but opinion variance decreases to represent agreement
(or, equivalently, \emph{consensus}~\cite{DeGroot1974}).

Null-polarization  We thus model opinions at each time, $t=\mathrm{pre}$ and
$t=\mathrm{post}$, as random variables drawn from a normal distribution with
mean $\mu_t$ and standard deviation $\sigma_t$.  we drop the time 

The null-polarization distributions are represented by 

in a simple, general way—i.e., a decrease in opinion variance but a static
opinion mean—we represent . In the general group polarization hypothesis, the
group is assumed to start out mostly agreeing in principle (i.e., mostly all
agree or disagree) but to different extremes. After discussion

Showing a replication is undecidable is sufficient to undermine it. However,
it does not tell us what is the probability that a particular experimental
design produces false positives, due to clipping or any other problem. 

first develop a mathematical model of estimate the best-case scenario false
positive rate for each experimental design,


\subsection{Mathematical Model of Null-Polarization Measurement Artifacts}

Latent opinions are modeled as normally-distributed random variables.  
This is a simple, sufficiently general way to represent group-level changes
in opinions as encoded in the general hypothesis of group polarization, that
likeminded groups tend to become more extreme on average as they discuss.
For time, $t$, then, we represent each participant $i$'s opinion as 
\begin{equation} 
  \omega_{i,t} \sim \mathcal{N}(\mu_t, \sigma_t).
  \label{eq:opinionDistribution} 
\end{equation} 
\noindent 

Group-level opinion change, then, can be represented as a change in opinion
mean, opinion variance, or both. Group polarization occurs 
when a group's average opinion increases over time. When likeminded people
interact, they tend to come to agree more over time—they \emph{find consensus}. 
By definition this means opinion variance decreases, 
whether the average changes or not.

We can define a null-polarization model, then, as one where the group's
average opinion stays constant, but the variance decreases: $\sigmapost
\leq \sigmapre$. Since the mean opinion is constant we just call it 
$\mu = \mupre = \mupost$. For conciseness, we define a null-polarization
model as a 3-tuple, $\nu = (\mu, \sigmapre, \sigmapost)$. 

\begin{center}
\vspace{1.5em}
[ TABLE \ref{tab:latent-variables} ABOUT HERE ]
\vspace{1.5em}
\end{center}

\subsubsection{Simulated opinion measurements}

We simulate the collection of an \emph{ordinal distribution} of \emph{ordinal
observations}, as was performed in a prototypical group polarization experiment 
by integrating the latent opinion probability density function, $f(\omega)$,
over each bin's corresponding range of opinion values. For example, consider an
ordinal scale that runs from -3 (strongly disagree) to +3 (strongly agree)
(as used by \textcite{Moscovici1969}, for example).
the frequency with which the ``2'' bin is chosen is just the total
area underneath the probability density curve between 1.5 and 2.5.

Formally, the probability density function is
\begin{equation}
  f(\omega) = 
    \frac{1}{\sigma \sqrt{2 \pi}} e^{-\frac{(\omega - \mu)^2}{2 \sigma^2}}.
  \label{eq:normal_pdf}
\end{equation}
\noindent
To set up our measurement simulation, we will represent
oridnal bin values as $b_i$ with $i=1,\ldots,B$—i.e., there are $B$ bins. 
To create $B$ bins there must be $B+1$ \emph{thresholds} that define what
range of latent opinions correspond to which ordinal bin value. 

\begin{center}
  \vspace{1.5em}
  [ TABLE \ref{tab:experiment-parameters} ABOUT HERE ]
  \vspace{1.5em}
\end{center}

All bins have a width of 1 in latent opinion space, except for the two most
extreme bins at opposite ends of the sentiment spectrum—this works because
bin values are always separated by 1. All continuous latent opinion values
$\omega < b_1 + \frac{1}{2} = \theta_1$ get mapped to the ordinal measurement
value $b_1$—so, $b_1$'s lower threshold is $\theta_0 = -\infty$.  Similarly,
all opinions $\omega > b_B - 0.5 = \theta_B$ are reported as the maximum bin
value, $b_B$, with an effective upper bound of $\theta_{B+1} = +\infty$.  We
can define all thresholds compactly as \cite{Liddell2018}
\begin{equation}
\theta_j = \begin{cases}
  -\infty         & \text{ if } j=0 \\
  \infty          & \text{ if } j=B \\
  b_1 + \frac{j + 1}{2} & \text{ otherwise.}
\end{cases}
\label{eq:thresholds}
\end{equation}
\noindent

We model the measurement of ordinal opinions by calculating the probability
density function over bins, written
\begin{equation}
  f[b_i] = \int_{\theta_{i - 1}}^{\theta_i} f(\omega) d\omega~,
  \label{eq:measurement-math}
\end{equation}
\noindent
switching to square brackets for the argument to $f$ now to emphasize that 
bins are discrete. With this ordinal measurement model we can caluclate the 
mean opinion of the simulated measurements like so
\begin{equation}
  \bar{o}_t = \frac{1}{B} \sum_{b=b_1}^{b_B} b \cdot f_t[b].
\end{equation}
\noindent

\begin{center}
\vspace{1.5em}
[ FIGURE \ref{fig:distros} ABOUT HERE ]
\vspace{1.5em}
\end{center}

Group polarization is the difference between the post-deliberation mean
observed opinion and the pre-deliberation one, 
which we can write as 
\begin{equation}
  g = \bar{o}_\mathrm{post} - \bar{o}_\mathrm{pre}.
  \label{eq:group-polarization}
\end{equation}
\noindent

\begin{center}
\vspace{1.5em}
[ TABLE \ref{tab:simulated-observations} ABOUT HERE ]
\vspace{1.5em}
\end{center}


\subsection{Root-finding Algorithm to Find Null-Polarization Models with Artifacts}


\subsection{Corpus Analysis to Determine Undecidable Replications}

To analyze published replications we need to expand our notation to index
individual \emph{experiments} with replications, and from which 
different journal articles, track and analyze it, 


\subsection{Evaluating Grouop Polarization Experimental Designs}


