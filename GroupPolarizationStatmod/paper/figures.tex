\section*{Figures}

\begin{figure}[ht]
  \caption{\textbf{Opinion measurement model and spurious group polarization.} 
  Latent model participant opinions are drawn from a normal distribution with the
  listed parameters (A,C). The pre-deliberation latent distribution has high
  variance, but appears polarized when measured via integration (A).  The latent
  distribution's variance is expected to decrease during deliberation (B) via
  consensus, but in this example the latent mean is static, 
$\mupre = \mupost$. However, when the post-deliberation distribution is measured,
the simulated observed mean increases, i.e., spurious
group polarization occurs (C).}
  \centering
  \includegraphics[width=0.95\textwidth]{Figures/Model/latent-ordinal-distros.pdf}
  \label{fig:distros}
\end{figure}


\begin{figure}
  \centering
    % \includegraphics[width=0.8\textwidth]{Figures/Model/data-model-flow.png}
  \includegraphics[width=\textwidth]{Figures/Model/SoftwareMap_Mashup.png}
  \caption{\textbf{Simulation and analysis pipeline}
  for finding when measurement artifacts.  
  Arrows show the flow of input and output data from parameter estimation.
  First, we used the web app (A) to find null-polarization parameters for 
  each experimental condition, $\nu_e$,
  that ordinal measurements distort to look like polarization
  (B—described in Algorithm~\ref{alg:hillclimbing}).
  We fit ordered probit models to simulated null-polarization outcomes,
  seeded by the $\nu$ found in the previous step (C). Analysis and 
  visualization routines finish the pipeline (D).}
  \label{fig:data-model-flow}
\end{figure}



