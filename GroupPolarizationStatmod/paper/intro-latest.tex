\subsection*{Abstract}\label{abstract}
\addcontentsline{toc}{subsection}{Abstract}

Echo chambers seem to cause radicalization—and radicalization is risky.
Radicalization undermines institutions by reducing their diversity, 
which in turn impairs their adaptability.  Social psychologists and others study
echo chambers experimentally using \emph{group polarization} experiments.
Researchers induce \emph{group polarization} by first asking experiment
participants their opinions on some topic, then grouping them together if
they agree in spirit, but not degree. The participants then discuss the topic
in their groups. Their opinions are then re-measured. Group polarization is
when the average group opinion increases. But there’s a fatal problem:
participants report opinions on ordinal scales, but this has never been
accounted for statistically. This can cause apparent group polarization in
observations even if the average latent opinion—the one in people’s
heads—doesn’t change, due to ceiling effects. Here we prove that across ten
journal articles spanning five decades that 54 of 57 reported polarization effects are plausibly false due to this artifact, and thus cannot be counted as replications. Our method can be used to evaluate any experimental design that measures latent change with ordinal scales. Given how destructive radicalization can be, we must get this right.


\section{Introduction}\label{introduction}

If an extremist’s opinion falls and it is measured with a Likert scale, will
it make a noise? The answer is often no~\cite{Liddell2018,Turner2020}. This
means a group’s opinions can appear to radicalize over time even when no real
shift occurs. The problem matters because Likert and other ordinal scales are
ubiquitous in research on opinion change and radicalization across the social
and cognitive sciences~\cite{Liddell2018}. One influential framework in this
area is group polarization—the idea that discussion among like-minded people
pushes opinions toward extremes, producing ``echo chambers.'' Cass Sunstein, a
Harvard Law professor and former White House administrator, has written
extensively on the topic~\cite{Sunstein2019}, claiming that group
polarization is a scientific law—a regular, quantitative relationship that
will always be observed when certain assumptions are
met~\cite{Cartwright1999}. Yet Roger Brown himself noted in Social Psychology
that group polarization ``does not occur with every group'' and that ``the
effect is not large.'' Given the influence of this theory and the hundreds of
claimed ``replications,'' it is essential to ask whether many of these findings
may instead be measurement artifacts—spurious effects created when the
``neutral'' point of the measurement scale fails to align with the group’s own
center of opinion.

The group polarization hypothesis is that discussion within a group of
likeminded people causes the group's average opinion to increase, or
\emph{polarize}~\cite{Brown1986,Brown2000}.
Group polarization experiments therefore attempt to
polarize groups as follows: first, participants report their opinion on
some topic or prompt chosen by the experimenters in advance. Second, 
participants are put into likeminded groups to
discuss the topic. Following discussion, participants again report their
opinions on the topic. Group polarization is defined as an increase in
the average opinion following discussion. If the group polarized, and we 
assume that people in the group found consensus—meaning their opinions agreed
more after the discussion than before—then initially moderate opinions must have
increased more than more extreme ones.

We are sad to report that 95\% of experimental reports of group polarization
across sixty published experimental conditions cannot be counted as
replications. We explain here how evidence of group polarization is undermined
by a basic measurement flaw that obscures opinion change among extremists.
This opens the possibility that what seems to be group polarization could just
be \emph{mere agreement}, with no change in mean opinion, if only increases in
opinion were measured but not decreases.  To evaluate whether the empirical
record reflects genuine opinion change or merely measurement artifacts, I
simulated published experiments under conditions of mere agreement. Each
simulation reproduced the core elements of published designs—pre-discussion
survey, group interaction, post-discussion survey—while holding the latent
mean opinion constant. Because the original datasets are lost and published
results rarely provide enough information to reconstruct them, we cannot know
whether any reported effect reflects a real shift or a measurement artifact.
When distinct latent processes can generate indistinguishable results, the
effect is non-identifiable—and no claim of replication can stand.

A relatively simple step could fix the problem, we believe: measure twice
before discussion. Before forming groups, experimenters could poll opinions
using a scale centered arbitrarily at 0, with varying strengths of disagree
or agree with negative and positive integers, respectively. 
Then after likeminded groups are
selected, measure group opinions again before they discuss, but now with the
scale centered on the average group opinion. To keep the same number of bins,
add more bins to express more extreme views and take away bins from
the opposite side as necessary.

Even if this measurement problem were fixed, equally severe problems remain
for group polarization research. First there seems to be no stable 
``opinions'' per
se—it is more accurate to say people construct opinions when they are
prompted or need them for some other reason, as revealed by respondents
answering questions differently if the framing or ordering of them changes,
and by the fact that people respond differently to the same question over
time. These are just a few of several potential sources of variance that were never accounted for, which is also known to generate false detections of change. 
Finally, because group polarization experimental design was never
standardized its research corpus cannot cumulatively or systematically tell 
us anything general about the world—Dorian
Cartrwright (1973) tried to alert group polarization researchers to this
problem half a century ago~\cite{Cartwright1973}. General scientific claims
of the existence and functioning of some scientifically induced phenomenon
only work if the evidentiary observations were collected from
\emph{commensurate} experiments, or in other words, only from experiments
that share the exact same causal structure as expressed in a formal or 
mechanistic causal model~\cite{Cartwright1999}.



\subsection{Group polarization research}

Group polarization research began in the 1960s, which grew rapidly with
focused government funding coupled with widespread
interest in what was then a surprising new science of
radicalization~\cite{Cartwright1971}. Dorian Cartwright (1973) reviewed the
first decade of group polarization research and found problems: he lamented
the ``disturbing fact'' that group polarization research has produced no
cumulative theoretical undersatnding of the effects of discussion on
group opinions~\cite{Cartwright1973}. Cartwright was somewhat dismayed, but
optimistically hoped that group polarizaiton research would eventually
produce ``an enduring\ldots more comprehensive paradigm'' \cite[p.
230]{Cartwright1973}.

Cartwright (1973) suggested that a ``vigorous search for alternative
formulations'' could help group polarization become more useful. More
researchers developed specific theories of group
polarization, with two becoming particularly influential: social comparisons
and persuasive arguments. The social comparisons hypothesis says group
polarization occurs because humans want to fit in, and so generally like to . Group polarization occurs, then, when a
group is composed mostly of extremists, with a sufficient number of people
with a weak opinion \cite{Myers1970; Myers1975}. Its main
“competitor” from the 1970s was the persuasive arguments theory, which
basically claimed that rhetoric was the strongest driver of group
polarization, not social comparisons \cite{Burnstein1973,Burnstein1975}.

Other notable hypotheses include the
self-categorization and social decisions scheme hypotheses.
Self-categorization augmented the social comparisons hypothesis with a
consideration of idiosyncratic beliefs people hold that are the result of
their own personal lived experience \cite{Abrams1990,Hogg1990}.
The social decisions hypothesis posits that contextual factors like
social network structure could be at least as important as the distribution
of opinions and . Noah Friedkin, a sociologist, notably published
the study with the least potential measurement artifacts for a rigorous study on group polarization, adopting the social
decisions scheme hypothesis, predicting that social connectivity within a
group could amplify group polarization \cite{Friedkin1999}.


\subsection{Study Corpus}

We selected ten representative journal articles with sixty total experimental
conditions among them. For a representative sample of group polarization
research across decades, we selected two articles for each of five article
categories: \emph{Social Comparisons}, \emph{Persuasive Arguments},
\emph{Self-Categorization}, \emph{Social Decisions}, and \emph{Politics}. The
first four of these categories reference the theoretical tradition that the
authors were trying to prove right—the fifth, \emph{Politics}, was the
subject matter of the studies, but the two studies took no clear side in the
group polarization theory wars. This resulted in a representative sample of
group polarization research over time, as well: one article is from 1969,
four are from the 1970s, . Information about our corpus is summarized below
(Table~\ref{tab:articles}).  Please see the Supplemental Information 


% \begin{table}[ht] 
  
%   \caption{\textbf{Journal articles in our corpus.}}
%   \label{tab:articles} 

%   \centering 
%   \rowcolors{2}{white}{gray!7}
%   \begin{tabular}{m{1.35in} m{2.45in}
%     m{1.3in} m{0.85in}}

%     \toprule

%     \textbf{Authors (Year)}& \textbf{Title} & \textbf{Journal} &
%     \textbf{Category} \\

%     \midrule 
%     Moscovici \& Zavalloni (1969) & The Group as a Polarizer of
%     Attitudes & \emph{JPSP} & Politics \\ 
%     Schkade, Sunstein, \& Hastie (2010) & When
%     Deliberation Produces Extremism & \emph{Critical~Review} & Politics \\

%     Bishop \& Myers (1970) & Discussion Effects on Racial Attitudes &
%     \emph{Science} & Social Comp's \\ 
%     Myers (1975) & Discussion-Induced Attitude Polarization & 
%     \emph{Human Relations} & Social Comp's \\

%     Burnstein \& Vinokur (1973) & Testing Two Classes of Theories About Group
%     Induced Shifts In Individual Choice & \emph{JESP} & Persuasive Arg's \\
%     Burnstein \& Vinokur (1975) & What a person thinks upon learning he has
%     chosen differently from others: Nice evidence for the
%     persuasive-arguments\ldots & \emph{JESP} & Persuasive Arg's \\

%     Abrams, et al., (1990) & Knowing What to Think by Knowing Who You Are:
%     Self‐Categorization and the Nature of Norm Formation\ldots & 
%     \emph{Brit. J. of Soc. Psych.} & Self-Categ'n \\
%     Hogg, Turner, \& Davidson (1990) & Polarized Norms and Social Frames of 
%     Reference: A Test of the Self-Categorization Theory of Group 
%     Polarization & \emph{B. \& Appl. Soc. Psych.} & Self-Categ'n \\
%     Krizan \& Baron (2007) & Group Polarization and Choice Dilemmas: How
%     Important is Self-Categorization? & \emph{Eur. J. of Soc. Psych.} &
%     Self-Categ'n \\

%     Friedkin (1999) & Choice Shifts \& Group Polarization & 
%     \emph{Amer. Soc.  Rev.} & Soc'l Decisions \\


% \bottomrule
% \end{tabular}
% \end{table}
\begin{table}[ht]
\centering
\caption{\textbf{Journal articles in our corpus.}}
\label{tab:articles}

\begin{tabular}{m{0.9in} m{1.35in} m{2.45in} m{1.3in}}
\toprule
\textbf{Category} & \textbf{Authors (Year)} & \textbf{Title} & \textbf{Journal} \\
\midrule

\rowcolor{gray!10}
\multirowcell{2}{Politics} 
  & Moscovici \& Zavalloni (1969) & The Group as a Polarizer of Attitudes & \emph{JPSP} \\
\rowcolor{gray!10}
  & Schkade, Sunstein, \& Hastie (2010) & When Deliberation Produces Extremism & \emph{Critical~Review} \\[0.2em]

\rowcolor{gray!5}
\multirowcell{2}{Social Comp's \vspace{0.7em}}
% \multirowcell{2}{Social Comp's}
  & Bishop \& Myers (1970) & Discussion Effects on Racial Attitudes & \emph{Science} \\[0.2em]
\rowcolor{gray!5}
  & Myers (1975) & Discussion-Induced Attitude Polarization & \emph{Human Relations} \\[0.3em]

\rowcolor{gray!10}
\multirowcell{2}{Persuasive Arg's}
  & Burnstein \& Vinokur (1973) & Testing Two Classes of Theories About Group Induced Shifts In Individual Choice & \emph{JESP} \\
\rowcolor{gray!10}
  & Burnstein \& Vinokur (1975) & What a person thinks upon learning he has chosen differently from others: Nice evidence for the persuasive-arguments\ldots & \emph{JESP} \\

\rowcolor{gray!5}
\multirowcell{3}{Self-Categ'n}
  & Abrams, et al., (1990) & Knowing What to Think by Knowing Who You Are: Self‐Categorization and the Nature of Norm Formation\ldots & \emph{Brit. J. of Soc. Psy.} \\
\rowcolor{gray!5}
  & Hogg, Turner, \& Davidson (1990) & Polarized Norms and Social Frames of Reference: A Test of the Self-Categorization Theory of Group Polarization & \emph{B. \& Appl. Soc. Psy.} \\
\rowcolor{gray!5}
  & Krizan \& Baron (2007) & Group Polarization and Choice Dilemmas: How Important is Self-Categorization? & \emph{Eur. J. of Soc. Psy.} \\

\rowcolor{gray!10}
\multirowcell{1}{Soc'l Decisions}
  & Friedkin (1999) & Choice Shifts \& Group Polarization & \emph{Amer. Soc. Rev.} \\

\bottomrule
\end{tabular}
\end{table}




\subsection{Measurement procedure}

In studies of group polarization, opinion measurements are typically ordinal, reflecting
ordered positions rather than measurable distances between them—
Moscovici and Zavalloni (1969) gave participants a seven-point scale, from “strongly disagree” (-3) to “strongly agree” (+3) to report their opinions.
Burnstein and Vinokur (1975) used a ten-point
scale from 1 to 10 representing “the lowest odds of success acceptable” to
try a certain behavior, where 1 means 10\%.  In these and every other survey-based group polarization study I know, the reported opinions on the ordinal scale—with discrete numbered bins—were treated as if they were continuous-valued, or in other words as if they were the internal mental opinion value itself, whatever that might be neurobiologically. 

Note how ordinal scales clip all latent opinions with a magnitude greater than the
magnitude of its extreme values. 
lead to false positives because ordinal scales cannot track changes in
extreme opinions beyond the ceiling of the measurement scale—this approach is
ubiquitous in social psychology and survey research, not at all limited to
group polarization.


