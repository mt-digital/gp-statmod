\section*{Tables}

\begin{table}[ht]
\centering
\caption{\textbf{Journal articles in our corpus.}}
\label{tab:articles}

\begin{tabular}{m{0.9in} m{1.35in} m{2.45in} m{1.3in}}
\toprule
\textbf{Category} & \textbf{Authors (Year)} & \textbf{Title} & \textbf{Journal} \\
\midrule

\rowcolor{gray!5}
\multirowcell{2}{Politics} 
                  & Moscovici \& Zavalloni (1969)~\cite{Moscovici1969} & The Group as a Polarizer of Attitudes & \emph{JPSP} \\
\rowcolor{gray!5}
                  & Schkade, Sunstein, \& Hastie (2010)~\cite{Schkade2010} & When Deliberation Produces Extremism & \emph{Critical~Review} \\[0.2em]

\multirowcell{2}{Social Comp's \vspace{0.7em}}
% \multirowcell{2}{Social Comp's}
                  & Myers \& Bishop (1970)~\cite{Myers1970} & Discussion Effects on Racial Attitudes & \emph{Science} \\[0.2em]
                  & Myers (1975)~\cite{Myers1975} & Discussion-Induced Attitude Polarization & \emph{Human Relations} \\[0.3em]

\rowcolor{gray!5}
\multirowcell{2}{Persuasive Arg's}
                  & Burnstein \& Vinokur (1973)~\cite{Burnstein1973} & Testing Two Classes of Theories About Group Induced Shifts In Individual Choice & \emph{JESP} \\
\rowcolor{gray!5}
                  & Burnstein \& Vinokur (1975)~\cite{Burnstein1975} & What a person thinks upon learning he has chosen differently from others: Nice evidence for the persuasive-arguments\ldots & \emph{JESP} \\

\multirowcell{3}{Self-Categ'n}
                  & Abrams, et al., (1990)~\cite{Abrams1990} & Knowing What to Think by Knowing Who You Are: Self‐Categorization and the Nature of Norm Formation\ldots & \emph{Brit. J. of Soc. Psy.} \\
                  & Hogg, Turner, \& Davidson (1990)~\cite{Hogg1990} & Polarized Norms and Social Frames of Reference: A Test of the Self-Categorization Theory of Group Polarization & \emph{B. \& Appl. Soc. Psy.} \\
                  & Krizan \& Baron (2007)~\cite{Krizan2007} & Group Polarization and Choice Dilemmas: How Important is Self-Categorization? & \emph{Eur. J. of Soc. Psy.} \\

\rowcolor{gray!5}
\multirowcell{1}{Soc'l Decisions}
                  & Friedkin (1999)~\cite{Friedkin1999} & Choice Shifts \& Group Polarization & \emph{Amer. Soc. Rev.} \\

\bottomrule
\end{tabular}
\end{table}


%%%%%%%%%%%%%%%%%%%
% TABLES OF VARIABLES IN METHOD
\begin{table}[h]
  \caption{\textbf{Latent psychological variables and distributions.}}
  \label{tab:latent-variables}
  \begin{tabular}{cll} 
  \toprule
    \textbf{Variable} & \textbf{Description} & \textbf{Values} \\
  \midrule  
    $\omega_{i,t}$ & Latent opinion of individual $i$ at time $t$ & $\mathbb{R}$ \\
    $\mu_t$ & Mean latent opinion at time $t$ (constant in null-polarization) & $\mathbb{R}$ \\
    $\sigma_t$ & Variance of observed opnions at time $t$ & $\mathbb{R}$ \\
  \bottomrule
  \end{tabular} 
\end{table}

\begin{table}[h]
  \caption{\textbf{Experimental design and numerical model parameters.}}
  \label{tab:experiment-parameters}
  \begin{tabular}{cll} 
  \toprule
    \textbf{Variable} & \textbf{Description} & \textbf{Values} \\
  \midrule  
    $B$ & Number of bins & $5,6,7,\ldots$ \\
    $\theta_b$ & The $B+1$ threshold values that separate bins & 
      e.g., $-\infty, -2.5, -1.5,\ldots,2.5, \infty$ \\
    $b$ & Bin index—there are $B+1$ to mark $B$ bins & $0,1,2,\ldots,B$ \\
  \bottomrule
  \end{tabular} 
\end{table}

\begin{table}[h]
  \caption{\textbf{Simulated observed variables.}}
  \label{tab:simulated-observations}
  \begin{tabular}{cll} 
  \toprule
    \textbf{Variable} & \textbf{Description} & \textbf{Values} \\
  \midrule  
    $\meanobst$ & Mean of all reported opinions (Equation \ref{eq:meanobs}) & $\mathbb{R}$ \\
    $g(\nu)$ & Spurious group opinion polarization for null-polarization params $\nu$
    (Equation \ref{eq:group-polarization})
        & $\mathbb{R}$ \\
    $d_{ei}$ & Cohen's $d$ for the $i^{\mathrm{th}}$ simulation of experiment $e$ & $\mathbb{R}$ \\
    $d*$ & Significance value in terms of Cohen's $d$, e.g., $d^* \geq 0.5$ as a
    "medium" effect\footnote{According to Cohen (1986)\cite{CohenBook1988}.} & $\mathbb{R}$ \\
    $\alpha_e,\alpha_s,\alpha_{\text{all}}$ & False positive rate
    for an experiment, $e$; study, $s$; or across \emph{all} experiments.
             & $[0, 1]$ \\
    FDR$_e$ & False detection rate: calculated using $\alpha_e$ to get chance experimental
    detection is false\footnote{See Equation~\ref{eq:fdr}}
        & $[0, 1]$ \\
  \bottomrule
  \end{tabular} 
\end{table}
